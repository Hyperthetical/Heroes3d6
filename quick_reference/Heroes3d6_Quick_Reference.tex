%Copyright (C) 2021-2024 Geoff Beck. This work is licensed under a Creative Commons Attribution-ShareAlike 4.0 License. To view a copy of this license visit: \url{https://creativecommons.org/licenses/by-sa/4.0/legalcode

\documentclass[a4paper,oneside,11pt]{article}
\usepackage{fullpage}
\usepackage{amsmath}
\usepackage{import}
\usepackage{tocbibind}
\usepackage[bookmarks=true,plainpages=false]{hyperref}
\usepackage{titletoc,titlecaps}
\usepackage{caption}
\usepackage[a4paper,margin=2cm]{geometry}
\usepackage[T1]{fontenc}
\usepackage{lmodern}
\usepackage{fancyhdr}
\usepackage{titling}
\title{\textbf{Heroes3D6\\Quick-Play Reference}}
\author{Geoff Beck}
\predate{\centering}
\date{}
\postdate{\vfill{\copyright 2021-2024 Geoff Beck. This work is licensed under a Creative Commons Attribution-ShareAlike 4.0 License. To view a copy of this license visit: \url{https://creativecommons.org/licenses/by-sa/4.0/legalcode}}\hfill}
\usepackage{lipsum}
\pagestyle{plain}

\usepackage[T1]{fontenc}
\usepackage{lmodern}

\newcommand{\dicediffbase}{11}
\newcommand{\dicecritlvl}{3}

\newcommand{\textlf}[1]{\textbf{\titlecap{#1}}}
\newcommand{\textlfirst}[1]{\textbf{\textit{\titlecap{#1}}}}

\begin{document}
\maketitle

\section{Natural Attributes}
These reflect various aspects of the character's personality and consist of: \textlf{Might}, \textlf{Cunning}, \textlf{Wit}, and \textlf{Resolve}.

\textlf{Might} is the character's aggression or assertiveness, \textlf{Cunning} is a measure of their craftiness, \textlf{Wit} is their alertness and attention to detail,  and \textlf{Resolve} their force of will and personal magnetism.

\subsection{Attribute Scores}
These vary between 0 and 2, with higher being better.

\section{Heroic Attributes}
\subsection{Heroism}
This is the characters capacity for heroic actions and is tracked in terms of available points. These can be spent on heroic efforts, which either guarantee success on a chosen action or add a level of \textlf{critical success}. Additionally, a character can use a point to benefit from a single \textlf{perk} of their choice for 1 round (no need to know or have it equipped to do this). If you had no \textlf{Heroism} remaining, then you regain one point when resting. Otherwise, the Game Master is free to reward clever moves, heroic behaviour, or good role-playing with a point of \textlf{Heroism}.

\section{Perks}
These represent talents gained while adventuring and can be purchased at the cost of experience.

\textlf{Perks} are divided into two categories: active and passive. \textlf{Active perks} alter the kind of actions your character can take. \textlf{Passive perks} provide their bonuses all the time. A single character can only have 3 active and 3 passive \textlf{perks} equipped at a time. They can still have \textlf{perks} purchased but these effects do not apply unless equipped. Changing equipped \textlf{perks} requires 1 hour of rest. There are slots on the character sheet for you to note your currently equipped \textlf{perks} in.

\section{Edge}
When rolling 3d6 with an \textlf{edge} bonus (\textlf{edge+}), a character rolls 4d6 and chooses the highest 3. In contrast, an \textlf{edge} penalty (\textlf{edge-}) makes the character roll 4d6 dice and choose the lowest 3.  The general pattern is: when rolling $n$d6 and having $m$ \textlf{edge} bonuses/penalties, you instead roll $(n+m)$d6 and choose the highest/lowest $n$ dice. 

\section{Difficulty Checks}
A \textlf{difficulty check} is made using a given \textlf{Attribute} or \textlf{Skill}. In order to succeed you must roll equal to or higher than the \textlf{Difficulty} on 3d6, adding the appropriate \textlf{Attribute} score.

\section{Opposed Rolls}
Each of the creatures contesting the roll adds the given score to the result of 3d6. The winner is the one with the highest total. Re-roll if it is a tie.


\section{Skills}

How to use a skill:
\begin{itemize}
\item
Roll 3d6, adding synergy attribute and level bonus/penalty (see below).
\item
If the roll equals or exceeds the \textlf{Difficulty}, the \textlf{skill} use is successful, and if it exceeds the \textlf{Difficulty} by $\dicecritlvl$ or more it is a \textlf{critical success}. Failing by $\dicecritlvl$ or more produces a \textlf{critical failure}. 
\item
These results have direct effect upon how well the desired task is achieved (see the skill rules for detailed consequences).
\end{itemize}

Skill levels: untrained (\textlf{edge-}), proficient (none), adept ($+1$), expert ($+2$), master ($+2$ and \textlf{edge+}).


\section{Combat}

\subsection{A round of combat}
A round of combat can be summarised simply as
\begin{enumerate}
	\item All combat participants declare intended actions (3 action points each)
	\item Each side in combat rolls 1d6
	\item Resolve actions for each side in order of the 1d6 roll
\end{enumerate}

\subsection{Combat actions}
A list of possible actions in combat is given below, the action point cost is given in brackets.

\subsubsection{Assist ally}
This costs 1 \textbf{AP} to prepare and allows you to spend 1 \textbf{RP} during this round to grant \textlf{edge+} to a single check made by an ally. You must be in the correct range to assist with the chosen check.

\subsubsection{Anticipate}
This costs 1 action point but grants the character a bonus reaction point that lasts until the next round ends.

\subsubsection{Walk (1)}
The character can transfer between adjacent combat areas (10 m of movement) and thus has range 1. This cannot trigger the \textlf{exploit weakness} reaction. 

\subsubsection{Run (1}
The character can move at range 2 but leaving an enemy's close-combat range trigger the \textlf{exploit weakness} reaction.  

\subsubsection{Attack (1)}
Make an opposed roll with \textlf{aim} versus the target's \textlf{defence} (you have \textlf{edge-} if not proficient with your weapon). If you win, make a \textlf{damage check} by rolling with your \textlf{power} versus target's \textlf{toughness}. Many \textlf{perks} can enhance or alter attacks. This costs 2 action points if using two weapons and each makes its roll separately.

\subsubsection{All-out attack (1)}  
Add $+1$ \textlf{aim} and \textlf{power} to your attacks this round. You can stack this bonus.

\subsubsection{Shove (1)}
Opposed \textlf{athletics} check with target within range 1 (who can use \textlf{coordination} instead). If you win they are moved a distance of 1, otherwise you trigger \textlf{exploit weakness} on \textlf{critical failure}. \textlf{critical success} results in the victim being \textlf{knocked down}. If multiple creatures \textlf{shove} the same target then they make a single roll with the highest \textlf{athletics} score from among them as well as \textlf{edge+} per creature beyond the first.

\subsubsection{Feint (1)}
Opposed attacker's \textlf{deceive} with defender's \textlf{awareness} (range 1). If the attacker succeeds they gain \textlf{edge+} on \textlf{aim}, otherwise they trigger \textlf{exploit weakness} on \textlf{critical failure}.

\subsubsection{Grapple (1)}
Opposed \textlf{athletics} check with target within range 1 (who can use \textlf{coordination} instead). If you win, they are \textlf{immobilised}, otherwise you trigger \textlf{exploit weakness} on \textlf{critical failure}. The victim can try to escape each turn with an opposed \textlf{athletics} check (costs 1 action point). It costs 1 action point each round after the first to maintain a grapple. 

\subsubsection{Trip (1)}
Trip attempts involve an \textlf{opposed roll} between the attacker and target using \textlf{coordination}. Should the attacker win, the target is \textlf{knocked down}. If this is \textlf{critical success} by X thresholds, the victim suffers \textlf{staggered} X. If the attacker \textlf{critically fails} a \textlf{trip}, then they trigger \textlf{exploit weakness}.

\subsubsection{Take cover}
If a creature is in a combat area that contains \textlf{cover}, they can shelter behind it for 1 \textbf{AP}. This grants a bonus to \textlf{defence}, as detailed on Table~\ref{tab:cover}, which is only lost if the creature leaves the combat area. In this state, a creature can also use the \textlf{Hide} action. 

\subsubsection{Taunt}
With cutting words, or obscene gestures, you shame and belittle the target. This costs 1 \textbf{AP}, has range 3, and the target must be able to understand the provocation in some way. Make an opposed check with your \textlf{perform skill} and the target's \textlf{resolve}. If you win, the target is \textlf{staggered} 1. \textlf{critical success} increases the level of the \textlf{stagger}. \textlf{critical failure} means you are \textlf{staggered} by your own ineptitude.

\subsubsection{Triage}
A creature can attempt to help its ally with some rapid medical attention. This is a \textlf{healing} check that costs 1 \textbf{AP} and can be used when in range 0 of the target. If successful, this reduces the severity of \textlf{wound} penalties for 10 minutes (it does not remove the \textlf{wounds}, just allows the target to ignore the penalty). The \textlf{difficulty} is 10 + \textlf{wound level}. This requires some medical supplies that are consumed during use (their cost and nature will vary by setting). 

\subsubsection{Dirty trick}
Sand in the eyes, sliced boot straps, smoke bombs, or any other underhanded trick to impair your foe's ability to fight. Provided you can perform a suitable bit of dirty fighting, spend 1 \textbf{AP} and make an \textlf{opposed roll} of your \textlf{slight of hand} against the \textlf{resolve} of a target within range 1. Success inflicts \textlf{staggered} 1, this increases by 1 for each \textlf{critical success} threshold. If the attacker \textlf{critically fails} a \textlf{dirty trick}, then the intended victim may use the \textlf{exploit weakness} reaction.

\subsubsection{Discard item}
For 0 \textbf{AP} a creature can discard an item it is currently holding. The item now sits in the combat area the discarder currently occupies and can be retrieved for 1 \textbf{AP} within range 0, this allows any enemies within range 0 to use the \textlf{exploit weakness} reaction.

\subsubsection{Diversion}
A character can create a diversion to distract other creatures. Make a \textlf{deception} check against the best hostile \textlf{awareness} within range 1, on a success the user gains \textlf{edge+} to \textlf{stealth} checks against these adversaries. In addition, diverted creatures cannot use the \textlf{exploit weakness reaction}. This can be used while \textlf{hidden} but ends the condition on a \textlf{critical failure}.

\subsubsection{Disarm (1)}
A disarm attempt consists of an \textlf{opposed roll}, made by the attacker and target within range 1, using \textlf{aim}. If the attacker wins, their target drops their weapon. \textlf{critical success} by X thresholds adds \textlf{staggered} X. The weapon lands somewhere in the same combat area as the victim and they can retrieve their weapon for 1 \textbf{AP} during their own turn but trigger \textlf{exploit weakness} for enemies within range 0. If the attacker \textlf{critically fails} a \textlf{disarm}, then they trigger \textlf{exploit weakness}.

\subsubsection{Defensive stance (1)}
The character gains \textlf{edge+} on \textlf{defence} for the next attack they suffer.

\subsubsection{Change weapon (1)}
A character can swap weapons (or other held items) for 1 action point per weapon being swapped. Consumables can be swapped to and used for just 1 action point.

\subsubsection{Hide}
A creature that has used \textlf{Take cover} can attempt to hide in it with a \textlf{stealth} check that costs 1 \textbf{AP}. The difficulty is set by the type of cover (see Table~\ref{tab:cover}), and if the character succeeds they are \textlf{hidden} (see Table~\ref{tab:conditions}). Enemies will still be aware the creature is hiding in a given combat area, \textlf{critical success} means this is no longer true (they know the creature is within a radius of 1) and allows the hider to move into an area within range 1 that also has some \textlf{cover}. 

\subsubsection{Intimidate}
A creature can try and scare their foe with a display of intimidating words, weapons, muscles, fangs, etc. This costs 1 \textbf{AP} and is an \textlf{intimidate} check opposed to the target's \textlf{resolve} (range 3). Success means the victim suffers from \textlf{Fear} (see Section~\ref{sec:special}). \textlf{critical success} increments the \textlf{edge-} level of the \textlf{fear}. On each subsequent round the victim may re-attempt the check to end the effect (no consequences for \textlf{critical success/Failure}). \textlf{critical failure} inflicts the penalty on the user for 1 round instead, demoralised by how laughable their threats are.

\subsection{Combat Attributes}
\begin{tabular}{lll}
\textlf{Aim} & = & \textlf{wit} \\ 
\textlf{defence} & = & \textlf{Cunning}\\
\textlf{Power} & = & Weapon Bonus + \textlf{might}\\
\textlf{Toughness} & = & armour value (or 8)\\
\textlf{endurance} & = & 2 + \textlf{resolve} \\
\end{tabular}


\subsection{Damage checks}
Are a difficulty check using the \textlf{Power} of the attack against a \textlf{difficulty} given by the target's \textlf{Toughness} (you have \textlf{edge-} if not proficient with your weapon). A successful \textlf{damage check} results in the target losing 1 \textlf{endurance} per point of \textlf{lethality} on the attack (default 1). If they run out of \textlf{endurance} then increment their \textlf{wound level} instead.

\subsubsection{Penetrating hits}
These are scored if a fighter fails a \textlf{defence} attempt by $\dicecritlvl$ or more. A \textlf{Penetrating Hit} grants \textlf{edge+} on the subsequent \textlf{damage checks}.

\subsubsection{Critical hits}
For every $\dicecritlvl$ the \textlf{damage check} exceeds the target's \textlf{toughness} by, the attack gains + 1 \textlf{Lethality}.


\subsection{Wound levels}
There are 4 \textlf{wound levels}: \textlf{wounded} (1), \textlf{badly wounded} (2), \textlf{mortally wounded} (3), and \textlf{dead} (4).
\begin{itemize}
	\item
	\textlf{Wounded} characters have $-1$ to all rolls. 
	\item
	A \textlf{Badly Wounded} causes \textlf{edge-} on all rolls.
	\item 
	A \textlf{Mortally Wounded} effect means the character cannot make actions and will die within 3 days without treatment.
	\item 
	A \textlf{dead} character is dead.
\end{itemize}

\subsection{Recovery and Healing}
\textlf{wound levels} and \textlf{endurance} are completely independent of each other. Thus, restoring \textlf{endurance} has no effect on a character's current \textlf{wound level} (and vice-versa). 
\begin{itemize}
	\item 
	1 \textlf{endurance} is restored by 1 hour rest or a hot meal.
	\item 
	All \textlf{endurance} is restored via a night's sleep.
	\item 
	The \textlf{leadership skill} can restore \textlf{endurance}
	\item 
	Restoring \textlf{endurance} has no effect on a target's \textlf{wound level}
\end{itemize}

\begin{itemize}
	\item
	Wounds can be healed via use of the \textlf{Healing} skill or by a non-player healer.
	\item
	A character heals from the \textlf{Wounded} effect after 3 days. \textlf{Badly Wounded} down-grades to \textlf{wounded} after 7 days.
\end{itemize}


\subsection{Reactions}

\subsubsection{Desperate effort}
This costs 1 reaction point and can be used when the character fails a check in combat (not including \textlf{skill} checks). This allows the character to re-roll the check, they must accept the result of this re-roll. This cannot be used if the character has any \textlf{edge} bonuses on the roll. 

\subsubsection{Exploit weakness}
This represents being able to take advantage of a vulnerable moment when a foe cannot defend themself. This costs 1 \textbf{RP} and allows the user to make a single close-combat \textlf{damage check} against the triggering enemy. Each character can use this only once per eligible target per turn.

\subsection{All-out defence}
The character can spend 1 reaction point in response to having to make a \textlf{Defence} check to get $+1$ on this roll.

\subsection{Changing actions}
Any character can spend a reaction point (before their actions are resolved) to change their actions from what they had previously declared. 

\subsection{Keep 'em down}
This costs 1 reaction point and can be used when an enemy within range 0 tries to rise from the \textlf{knocked down} state. Make an \textlf{opposed roll} against the \textlf{knocked down} enemy using \textlf{athletics}. If the reactor wins then the target remains \textlf{knocked down}. 



\section{Terrain and Cover}
Moving through \textlf{Rough} or \textlf{Dangerous} terrain has the following consequences:
\begin{itemize}
\item
\textlf{Rough} terrain: \textlf{running} requires \textlf{coordination} check (\textlf{difficulty} range 10 to 14) or become \textlf{knocked down}.
\item 
\textlf{Dangerous} terrain: Any character entering, occupying, or moving through this terrain must pass an \textlf{coordination} check of difficulty between 12 and 17 (depending on how dangerous the terrain is) or suffer damage with \textlf{lethality} 1 $+1$ per level of \textlf{critical failure}. All \textlf{move}-type actions cost 1 extra \textbf{AP} in this terrain.
\end{itemize}
After using a \textlf{take cover} action the character receives a bonus to their \textlf{defence} checks.
\begin{table}[ht]
	\centering
	\caption{Cover table}
	\label{tab:cover}
	\begin{tabular}{|l|l|l|}
		\hline
		Cover type & Defence bonus & Hide difficulty\\ [0.5ex]
		\hline
		None & - & 16 \\
		Soft & $+1$ & 13\\
		Medium & \textlf{edge+} & 11\\
		Heavy & \textlf{edge++} & 9\\
		\hline
	\end{tabular}
\end{table}

\section{Useful Tables}

\begin{table}
\centering
\caption{Condition Table}
\label{tab:conditions}
\begin{tabular}{|l|l|}
\hline
Condition & Effect \\
\hline
\textlf{Vulnerable} & Next \textlf{damage check} against victim has \textlf{edge+}\\
\textlf{Staggered} X & \textlf{edge-} on next X rolls\\
\textlf{Dazed} & Cannot spend \textbf{RP} \\
\textlf{Hidden} & Cannot be seen, can be detected with \textlf{awareness} vs \textlf{stealth} \\
\textlf{Fear} & \textlf{Edge-} on \textlf{Aim} and \textlf{skills} \\
\textlf{Hardened} X & Next X instances of damage have \textlf{Lethality} 1 \\
\textlf{Blind} & \textlf{defence} and \textlf{Aim} have \textlf{edge--} \\
\textlf{Bleeding} & \textlf{lethality} 1 damage each round. 2 action points to end effect. \\
\textlf{Burning} & \textlf{lethality} 1 damage each round. 2 action points to end effect. \\
\textlf{Knocked Down} & \textlf{edge-} on \textlf{defence} until it can stand up \\
\textlf{slowed} & The victim's movement actions cost 1 extra \textbf{AP}. \\
\textlf{Immobilised} & No move actions and \textlf{edge-} on \textlf{defence}\\
\textlf{Stunned} & Victim has 1 fewer \textbf{AP} \\
\textlf{Cursed} & \textlf{Critical success} on any roll is reduced by 1 level \\
\textlf{Terror} & Can only cower in fear \\
\hline
\end{tabular}
\end{table}

\begin{table}[ht!]
	\centering
	\caption{Skills (synergy attributes in brackets)}
	\begin{tabular}{|l|l|l|}
		\hline
		General & Social & Knowledge\\ [0.5ex]
		\hline
		Athletics (M)  &  Deceive (C)  &  Animals (W) \\
		Awareness (W)  &  Disguise (C)  &  Arcana (W) \\
		Coordination (C) &  Insight (R)  &  History (W) \\
		Healing (R)   &  Intimidation (M)  &  Plants (W) \\
		Mechanical (W) &  Leadership (R)  &  Religion (W) \\
		Pilot (W) &  Perform (R)  &  \\
		Slight of Hand (C)  &  Persuade (R)  &  \\
		Stealth (C) & & \\
		Survival (C)  &  &  \\
		\hline
	\end{tabular}
\end{table}

\end{document}